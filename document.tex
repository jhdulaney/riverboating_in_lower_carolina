\documentclass[11pt, a5paper]{book}

% This is a transcription done by Mairi Dulaney of Seattle, WA

% chktex-file 18

\usepackage{caption}
\usepackage[english]{babel}
\usepackage[T1]{fontenc}
\usepackage[]{geometry}
\usepackage{textcomp}
\usepackage{gensymb}
\usepackage{graphicx}
\graphicspath{{images/}}
\usepackage[utf8]{inputenc}
\usepackage[round,sort,comma]{natbib}
\usepackage{titlesec}

\usepackage[activate={true,nocompatibility},final,
  tracking=true,kerning=true,factor=1100,stretch=10,shrink=10]{microtype}

\geometry{a5paper}
\nonfrenchspacing
\titleformat{\chapter}[display]
{\bfseries\Large}
{\centering
  \textsc{Chapter} \thechapter.} % label
{0.5ex}
{
  \vspace{1ex}
  \centering
  \scshape
}
[\vspace{-1ex}]

\titleformat{\section}[display]{\centering\bf}{\enspace}{0em}{}

\overfullrule=2cm

\renewcommand{\thefootnote}{\fnsymbol{footnote}}


\begin{document}

\renewcommand{\thechapter}{\Roman{chapter}}
\title{Riverboating in Lower Carolina}
\author{F. Roy Johnson}
\date{1977}
\bibliographystyle{alpha}
\maketitle


\chapter{By Paddle, Ore, and Pole}

\section{Early Boats of Burden}

\textsc{Although} the dugout canoe, or log boat, has disappeared from
the rivers, creeks, and other waters of North Carolina's southeastern
counties, the watersheds of the Cape Fear, Black, and Northeast
rivers, it must be remembered as the common and more important vehicle
of travel and commerce of the early settlers of the region.  And to a
diminishing extent this versatile vehicle continued in common service
far into the nineteenth century.  As late as the 1950's a few
serviceable old moss cloaked log boats reposed in the South River
wilds of Bladen and Sampson counties and at remote landings of the
upper Waccamaw River in Columbus County.\par

The numerous waterways of the southeastern area --- the sounds, the
rivers, and the creeks --- were well suited for small boats.  These
avenues grew into the great channels of commerce and retained their
commanding position for nearly two centuries, until the railroads
became common and transportation needs moved westward.\par

Although much of the early travel was overland by horseback and by
foot on trails little better than Indian paths, virtually all the
commerce was by water.  However, by the Revolution, overland traffic
had increased; for the industrious settlers had built roads,
constructed causeways across swamps, bridged streams, and established
ferries.  This lessened little the water freightage.  And up and down
the waterways there plied a great variety of craft loaded with
commodities.\par

Plantations, trading places, villages, and towns sprang up along the
streams; and the growth of the towns was determined to a large extent
by the volume of this commerce.  Chief of the towns were Fayetteville
at the head of navigation on the Cape Fear River, which funneled goods
down stream from the back country and received imported commodities,
and Wilmington, which exported and imported goods for the watersheds
of the Cape Fear River and her tributaries.\par

Not so fortunate was the town of Lisbon at the head of the Black River
in Sampson County.  Its merchants prospered and the town grew and
flourished on pole boat traffic.  But after steamboats came it went
neglected because of the river's shallows; and it dried up completely
when bypassed by the railroads.\par

\section{The Canoe and Periauger}

The canoe's larger companion, the periauger, was a boat of burden
which served in the early commerce of the southeastern counties.  And
as the volume of commerce increased it was followed by the larger flat
boat, or pole boat.\par

Writing in 1737, a time when settlers were pouring into the
southeastern area, Dr.\ John Brickell told of the building and use of
the canoe and the periauger.  The canoe, eh said, was made out of one
piece of timber and commonly of cypress, the durable giant of the
coastal swamp forests.  A log of proper thickness and length was cut
from the felled tree, and craftsmen shaped the log like a boat and
then hollowed it out.  Sometimes the boat was made wider by splitting
it down the middle and installing boards in its bottom; and this boat
was called the periauger.  Then the canoe or periauger was fitted with
masts for sailing or oars and paddles, ``according to \ldots size and
bigness.''\par

Some of the larger of these craft were ``capable of carrying forty or
fifty Barrels of Pitch and Tar.''  The planters also found the
periauger useful as ferries and for transporting goods from one
plantation to another.  The design of this great log boat made it so
efficient that ``no Vessel of the same Burthen made after the European
manner'' could outsail one.\citep[260-261]{brickellj}\par

William Robinson, a Sampson County planter who lived at the Delta on
Black River about sixty miles from Wilmington by water, was using a
large cypress log boat in regular commerce with that seaport town in
the late eighteenth century.  He manned her with six black oarsmen.
And late in the nineteenth century William's great-grandson Scott
Robinson was running a turpentine distillery at the Delta and
freighting naval stores to Wilmington on a fifty foot by twelve foot
flat poled by four blacks.  \citep{shawg}\par

The canoe was made in many sizes.  The smaller might carry no more
than two or three men while the larger could carry two or three
horses.  They seem to have continued quite numerous throughout the
eighteenth century.  During the spring of 1762 a group of Moravians
proceeding up the Cape Fear River watched their boats by night for
fear of Negroes who ``go about in small canoes, stealing where they
can.'' \citep[I,~261]{friesal}.  And in 1775 Janet Schaw, a Scotch
woman sojourning in North Carolina, saw more than one hundred people of
the lower class in canoes come to a funeral on the Northeast
River.\citep[171]{schawj}\par

The larger boats used sails to great advantage, except on narrow and
croked streams where oars and poles were needed.  Tides favored all
kinds of craft upon the lower rivers, and freshet waters often aided
navigation upon their headwaters.\par

\section{Eighteenth Century Row Boats}

Late in the eighteenth century large \textit{row boats} were the
carriers of produce on the Cape Fear River between Cross Creek, near
present Fayetteville, and the Wilmington area \citep[280-281]{schawj}.
The planter near Wilmington manned such boats with slaves.  For travel
one such planter had ``a very fine boat with an awning to prevent the
heat and six stout Negroes in neat uniforms to row her down, which with
the assistance of the tide was performed in a very short time''
\citep[177]{schawj}.\par

As the volume of commerce increased, so did the size of the boats.
This led to the construction of the flat boat, or \textit{flat}, so
called because of its flat bottom and squared sides, which was built
of boards.  Being of shallow draught it could be used on small
streams.\par

Navigation of the streams often was beset with problems.  Hugh
Meredith, who about 1730 had come to North Carolina from Pennsylvania
to settle, wrote that the channel of the Northeast Cape Fear River was
impeded by the ``multitude of Logs that lie in it, part of them fast in
the Sand, with great Snags or Limbs, and sometimes either End the
Middle quite above, or but little beneath the Surface; and in some
places we saw whole Heaps jumbled together, almost from Side to Side,
and so firm that they are immovable, being sound, heavy, fast, and
deep in the Sand.'' (Meredith, Hugh, 21-22).\par % chktex 8

Soon after the Revolution there was a burst of enthusiasm for
improvement of waterways for navigation.  Some progress was made in
making passable the channels of the main streams.  And by the opening
of the nineteenth century virtually every little stream was cleared of
obstructions so as to accomodate small boats and flats, giving many
farmers located off the large streams access to water
transportation.\par

\section{The Pole Boat}

South (Black) River, a tributary of Black River, which rises in
Harnett County, was an important avenue of commerce for farmers of
eastern Cumberland County.  They ran their rafts and pole boats upon
it until late in the nineteenth century when railroads provided an
alternate means of transportation.\par

Caleb C. Bullard, born in 1859 in Beaverdam Township, recalled that
naval stores were carried down from this area to Wilmington on flat
boats.  John A. Oates in his \textit{The Story of Fayetteville} quotes
Bullard as saying, ``There was no trouble in getting down the river,
except to keep the flat boat guided; there was no steam power on South
River.  It took usually about four days for the trip down and six days
for the trip back, about 100 miles each way.  In Wilmington, they
would secure, usually, a tug boat to bring them to the mouth of South
River and from there up they would use poles twenty feet long or more,
to push the flat boat on up the river.  They would have six or eight
men to do this.''\par

Naval stores were the chief source of income for the Cumberland
farmers until some time after the Civil War when they began to raise
cotton.  In the depressed economy they found little inducement to grow
more than enough food to feed their families and livestock
\citep[432]{oatesja}.\par

\section{The Large Flat of Tidewater}

In tidewater no other power than the tide was needed to move the large
flat.  It was tied up against the tide, and when the tide turned it
floated like a raft with it.  Large flats, some carrying as much as 100
to 200 barrels of turpentine, tar, and pitch, were in use upon the
lower Cape Fear before the Revolution \citep[184-185]{schawj}, but
smaller craft were common above tidewater until the coming of the
steamboat which could tow them upstream against the current.\par

Some idea of flat boat sizes early in the nineteenth century may be
obtained from an advertisement in the November 21, 1816
\textit{Fayetteville American}.  An advertiser was seeking to let a
contract for construction of four flat boats, two 48 feet by ten feet
and two 30 feet by six feet.  They were to be made of heavy timber and
sides banded with strap iron \citep[11-21-16]{fa}.\par

Midway through the nineteenth century tide powered flats were carrying
300 to 500 barrels of turpentine or an equal amount of other
cargo.\footnote{The Wilmington Journal of 1851, under ``Marine
  Intelligence'', provides a record of pole boat activity.}  For
example, flat boat \textit{J.~L.~Cassidy} from Lyons Landing on the
lower Black River arrived in Wilmington on April 11, 1851 with 326
barrels of rosin, 64 barrels of tar, and 64 barrels of turpentine.  By
comparison, on January 31 the flat boat \textit{Sopha}, with Man
William in charge, from Beatty's Bridge on the Black River 50 miles
above Wilmington and above tidewater, arrived with 38 bales of hay,
five bales of fodder, and four sheafs of oats.  It's cargo was
comparable to about ten barrels of turpentine.\par

Meanwhile, the large tide powered flat boat was hauling much of the
cargo on the lower Cape Fear, Black, and Northeast rivers.  White
Hall, on the Cape Fear about 18 miles below present Elizabethtown and
54 miles above Wilmington became an important receiving and trading
center.  With some assistance from poles the great flats were able to
make their way this far up with no great difficulty.  Those going
higher up mostly were towed by steamboats.  Some of these, especially
those owned by the steamboat companies, were towed back down stream;
and these were called \textit{tow boats}.\par

Often the large flat boat was given a name which honored its owner or
some other person and put in charge of a trusted slave.  In 1851 one
finds Man Dick in charge of \textit{Uncle Sam}; Man Sandy, the
\textit{David Lewis}, Man Bob, the \textit{Stevenson}; Man Wilson, the
\textit{General Taylor}; Man Jack, the \textit{Dried Apple}; Man
Daniel, the \textit{James Ellis}; Man Dick, the
\textit{J.~L.~Cassidy}.  Other big flats were the \textit{David Reed},
the \textit{Jackson}, the \textit{General Cass}, \textit{Macmillan's
  Boat}, and \textit{Robinson's Boat}.\par

\section{The Steam Flat}

The steam flat followed the pole and row boat and small flat into the
narrow and shallow headwaters of the Black and Northeast rivers and
large creeks to compete with rafts in bringing down naval stores from
the developing country.  Although some of the early steamboats on the
Cape Fear River were little more than steam-powered paddle-wheel
flats, the steam flat as a distinctively different kind of craft did
not appear until after the Civil War when the entire southeastern area
of North Carolina was faced with the need for economical
transportation.\par

In 1886 ten of these vehicles of burden were transporting naval stores
and other commodities from the back country of Bladen, Brunswick,
Pender, Sampson, and Onslow counties to Wilmington.  Two of these were
operated on the Great Coharie, a shallow tributary of Black River up
to 100 miles from Wilmington.  \citep[8-13-86]{ws}\par







\begin{thebibliography}{99}

\section{Books, Pamphlets, Theses, Papers}

\bibitem[\protect\citeauthoryear{Brickel,~J.}{1968}]{brickellj}
  Brickell, John.  \emph{The Natural History of North Carolina},
  1968 reprint of 1737, Johnson Publishing Co., Murfreesboro, N.~C.

\bibitem[\protect\citeauthoryear{Clark,~W.}{1861-1865}]{clarkw}
  Clark, Waler.  \emph{North Carolina Regiments, 1861-1865}, % chktex 8
  5 volumes, Volume IV. % chktex 13

\bibitem[\protect\citeauthoryear{Crittenden,~C.~C.}{April 1931}]{crittenden}
  Crittenden, Charles Christopher.  ``Inland Navigation in North Carolina, 1763-1789'', % chktex 8
  \emph{North Carolina Historical Revue}, VIII (April 1931), 145-154. % chktex 8

\bibitem[\protect\citeauthoryear{Evans,~W.~McKee}{1966}]{evansw}
  Evans, W. McKee.  \emph{Ballots and Fence Rails, Reconstruction on the Lower Cape Fear}.
  Chapel Hill, University of North Carolina Press, 1966.

\bibitem[\protect\citeauthoryear{Fries,~A.~L. and others}{}]{friesal}
  Fries, A.~L.~and others.  \emph{Records of the Moravians in N.~C.},
  12 volumes, Vol. I.

\bibitem[\protect\citeauthoryear{Johnson,~G.~G.}{1937}]{johnsongg}
  Johnson, Guion Griffis.  \emph{Ante-Bellum North Carolina, A social History}.
  Chapel Hill, University of North Carolina Press, 1937.

\bibitem[\protect\citeauthoryear{Lee,~L}{1937}]{leel}
  Lee, Lawrence.  \emph{The Lower Cape Fear in Colonial Days}.
  Chapel Hill, University of North Carolina Press, 1937.

\bibitem[\protect\citeauthoryear{Lefler,~H.~T.}{1956}]{leflerht}
  Lefler, Hugh Talmadge.  \emph{History of North Carolina}, 4 volumes.  New
  York, Lewis Publishing Company, 1956.

\bibitem[\protect\citeauthoryear{Oates,~J.~A.}{1950}]{oatesja}
  Oates, John A.  \emph{The Story of Fayetteville and the Upper Cape Fear}.
  Charlotte, The Dowd press, 1950.

\bibitem[\protect\citeauthoryear{Olmstead,~F.~L.}{1904}]{olmsteadfl}
  Olmstead, Frederic Law.  \emph{A Journey in the Seaboard Slave States, in the
    Years 1853-1854, With Remarks on Their Economy}, 2 volumes.  New York and % chktex 8
  London, The Knickerbocker Press, 1904.

\bibitem[\protect\citeauthoryear{Powell,~W.~S.}{1968}]{powellws}
  Powell, William S.  \emph{The North Carolina Gazeteer}.  Chapel Hill,
  University of North Carolina Press, 1968.

\bibitem[\protect\citeauthoryear{Reilly,~J.~S.}{1884}]{reillyjs}
  Reilly, J.~S.  \emph{Wilmington---Past, Present, and Future}.  Wilmington,
  1884.

\bibitem[\protect\citeauthoryear{Roberson,~E.~E.}{}]{robersonee}
  Roberson, Elizabeth Ellis.  \emph{Diary}.  Elizabethtown, Bladen County
  Historical Association.

\bibitem[\protect\citeauthoryear{Schaw,~J.}{1939}]{schawj}
  Schaw, Janet.  \emph{Journal of a Lady of Quality}.  New Haven, Yale
  University Press, 1939.  Spartanburg, S.~C.~reprint.

\bibitem[\protect\citeauthoryear{Sharpe,~B.}{1954, 1958, 1961, 1965}]{sharpeb}
  Sharpe, Bill.  \emph{A New Geography of North Carolina}, 4 volumes.  Raleigh,
  Edwards \& Broughton Co., 1954, 1958, 1961, 1965.

\bibitem[\protect\citeauthoryear{Sloan, T.~H.}{1971}]{sloanth}
  Sloan, Thomas H.  \emph{Inland Steam Navigation in North Carolina 1818-1900}. % chktex 8
  Unpublished master's thesis, East Carolina University, Greenville, N.~C., 1971.

\bibitem[\protect\citeauthoryear{Sprunt, J.}{1916}]{spruntj1916}
  Sprunt, James.  \emph{Chronicles of the Cape Fear River 1660-1916}, Second % chktex 8
  Edition.  Raleigh, Edwards \& Broughton Printing Co., 1916.

\bibitem[\protect\citeauthoryear{Sprunt, J.}{1896}]{spruntj1896}
  Sprunt, James.  \emph{Tales and Traditions of the Lower Cape Fear 1661-1896}. % chktex 8
  Wilmington, LeGwin Brothers, 1896.

\bibitem[\protect\citeauthoryear{Turlington, S.~W}{1933}]{turlingtons}
  Turlington, Sarah Woodall.  \emph{Steam Navigation in North Carolina Prior
    to 1860}.  Unpublished master's thesis, University of North Carolina, Chapel
  Hill, 1933.

\bibitem[\protect\citeauthoryear{Whedbee, W.~L.}{}]{whedbeewl}
  Whedbee, W.~L.  \emph{Waterway Transportation in Eastern North Carolina}.
  Typescript, University of North Carolina Library.


\section{Newspapers and Periodicals}

\bibitem[\protect\citeauthoryear{BJ}{}]{bj}
  BJ---\emph{Bladen Journal}, Elizabethtown, N.~C.

\bibitem[\protect\citeauthoryear{CC}{}]{cc}
  CC---\emph{Clinton Caucasian} Clinton

\bibitem[\protect\citeauthoryear{FA}{}]{fa}
  FA---\emph{Fayetteville American}

\bibitem[\protect\citeauthoryear{FDC}{}]{fdc}
  FDC---\emph{Fayetteville Daily Courier}

\bibitem[\protect\citeauthoryear{FNCA}{}]{fnca}
  FNCA---\emph{Fayetteville North Carolina Argus}

\bibitem[\protect\citeauthoryear{FO}{}]{fo}
  FO---\emph{Fayetteville Observer}

\bibitem[\protect\citeauthoryear{FWM}{}]{fwm}
  FWM---\emph{Free Wilmington Magazine}

\bibitem[\protect\citeauthoryear{HR}{}]{hr}
  HR---\emph{Hillsboro Record}

\bibitem[\protect\citeauthoryear{WS}{}]{ws}
  WS---\emph{Wilmington Weekly Star}

\section{Traditional Sources}

\bibitem[\protect\citeauthoryear{Shaw, G.}{}]{shawg}
  Shaw, Graham, Atkinson, N.~C., native of Sampson County, N.~C.

\end{thebibliography}

\end{document}
