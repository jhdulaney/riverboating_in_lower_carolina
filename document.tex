\documentclass[11pt, a5paper]{book}

% This is a transcription done by Mairi Dulaney of Seattle, WA

% chktex-file 18

\usepackage{caption}
\usepackage[english]{babel}
\usepackage[T1]{fontenc}
\usepackage[]{geometry}
\usepackage{textcomp}
\usepackage{gensymb}
\usepackage{graphicx}
\graphicspath{{images/}}
\usepackage[utf8]{inputenc}
\usepackage[round,sort,comma]{natbib}
\usepackage{titlesec}

\usepackage[activate={true,nocompatibility},final,
  tracking=true,kerning=true,factor=1100,stretch=10,shrink=10]{microtype}

\geometry{a5paper}
\nonfrenchspacing
\titleformat{\chapter}[display]
{\bfseries\Large}
{\centering
  \textsc{Chapter} \thechapter.} % label
{0.5ex}
{
  \vspace{1ex}
  \centering
  \scshape
}
[\vspace{-1ex}]

\titleformat{\section}[display]{\centering\bf}{\enspace}{0em}{}

\overfullrule=2cm


\begin{document}

\renewcommand{\thechapter}{\Roman{chapter}}
\title{Riverboating in Lower Carolina}
\author{F. Roy Johnson}
\date{1977}
\bibliographystyle{alpha}
\maketitle


\chapter{By Paddle, Ore, and Pole}

\section{Early Boats of Burden}

\textsc{Although} the dugout canoe, or log boat, has disappeared from
the rivers, creeks, and other waters of North Carolina's southeastern
counties, the watersheds of the Cape Fear, Black, and Northeast
rivers, it must be remembered as the common and more important vehicle
of travel and commerce of the early settlers of the region.  And to a
diminishing extent this versatile vehicle continued in common service
far into the nineteenth century.  As late as the 1950's a few
serviceable old moss cloaked log boats reposed in the South River
wilds of Bladen and Sampson counties and at remote landings of the
upper Waccamaw River in Columbus County.\par

The numerous waterways of the southeastern area --- the sounds, the
rivers, and the creeks --- were well suited for small boats.  These
avenues grew into the great channels of commerce and retained their
commanding position for nearly two centuries, until the railroads
became common and transportation needs moved westward.\par

Although much of the early travel was overland by horseback and by
foot on trails little better than Indian paths, virtually all the
commerce was by water.  However, by the Revolution, overland traffic
had increased; for the industrious settlers had built roads,
constructed causeways across swamps, bridged streams, and established
ferries.  This lessened little the water freightage.  And up and down
the waterways there plied a great variety of craft loaded with
commodities.\par

Plantations, trading places, villages, and towns sprang up along the
streams; and the growth of the towns was determined to a large extent
by the volume of this commerce.  Chief of the towns were Fayetteville
at the head of navigation on the Cape Fear River, which funneled goods
down stream from the back country and received imported commodities,
and Wilmington, which exported and imported goods for the watersheds
of the Cape Fear River and her tributaries.\par

Not so fortunate was the town of Lisbon at the head of the Black River
in Sampson County.  Its merchants prospered and the town grew and
flourished on pole boat traffic.  But after steamboats came it went
neglected because of the river's shallows; and it dried up completely
when bypassed by the railroads.\par

\section{The Canoe and Periauger}

The canoe's larger companion, the periauger, was a boat of burden
which served in the early commerce of the southeastern counties.  And
as the volume of commerce increased it was followed by the larger flat
boat, or pole boat.\par

Writing in 1737, a time when settlers were pouring into the
southeastern area, Dr.\ John Brickell told of the building and use of
the canoe and the periauger.  The canoe, eh said, was made out of one
piece of timber and commonly of cypress, the durable giant of the
coastal swamp forests.  A log of proper thickness and length was cut
from the felled tree, and craftsmen shaped the log like a boat and
then hollowed it out.  Sometimes the boat was made wider by splitting
it down the middle and installing boards in its bottom; and this boat
was called the periauger.  Then the canoe or periauger was fitted with
masts for sailing or oars and paddles, ``according to \ldots size and
bigness.''\par

Some of the larger of these craft were ``capable of carrying forty or
fifty Barrels of Pitch and Tar.''  The planters also found the
periauger useful as ferries and for transporting goods from one
plantation to another.  The design of this great log boat made it so
efficient that ``no Vessel of the same Burthen made after the European
manner'' could outsail one.\citep[260-261]{brickellj}\par

William Robinson, a Sampson County planter who lived at the Delta on
Black River about sixty miles from Wilmington by water, was using a
large cypress log boat in regular commerce with that seaport town in
the late eighteenth century.  He manned her with six black oarsmen.
And late in the nineteenth century William's great-grandson Scott
Robinson was running a turpentine distillery at the Delta and
freighting naval stores to Wilmington on a fifty foot by twelve foot
flat poled by four blacks.  \citep{shawg}\par

The canoe was made in many sizes.  The smaller might carry no more
than two or three men while the larger could carry two or three
horses.  They seem to have continued quite numerous throughout the
eighteenth century.  During the spring of 1762 a group of Moravians
proceeding up the Cape Fear River watched their boats by night for
fear of Negroes who ``go about in small canoes, stealing where they
can.'' \citep[I,~261]{friesal}.  And in 1775 Janet Schaw, a Scotch
woman sojourning in North Carolina, saw more than one hundred people of
the lower class in canoes come to a funeral on the Northeast
River.\citep[171]{schawj}\par

The larger boats used sails to great advantage, except on narrow and
croked streams where oars and poles were needed.  Tides favored all
kinds of craft upon the lower rivers, and freshet waters often aided
navigation upon their headwaters.\par







\begin{thebibliography}{99}

\section{Books, Pamphlets, Theses, Papers}

\bibitem[\protect\citeauthoryear{Brickel,~J.}{1968}]{brickellj}
  Brickell, John.  \emph{The Natural History of North Carolina},
  1968 reprint of 1737, Johnson Publishing Co., Murfreesboro, N.~C.

\bibitem[\protect\citeauthoryear{Clark,~W.}{1861-1865}]{clarkw}
  Clark, Waler.  \emph{North Carolina Regiments, 1861-1865},
  5 volumes, Volume IV.

\bibitem[\protect\citeauthoryear{Crittenden,~C.~C.}{April 1931}]{crittenden}
  Crittenden, Charles Christopher.  ``Inland Navigation in North Carolina, 1763-1789'',
  \emph{North Carolina Historical Revue}, VIII (April 1931), 145-154.

\bibitem[\protect\citeauthoryear{Evans,~W.~McKee}{1966}]{evansw}
  Evans, W. McKee.  \emph{Ballots and Fence Rails, Reconstruction on the Lower Cape Fear}.
  Chapel Hill, University of North Carolina Press, 1966.

\bibitem[\protect\citeauthoryear{Fries,~A.~L. and others}{}]{friesal}
  Fries, A.~L.~and others.  \emph{Records of the Moravians in N.~C.},
  12 volumes, Vol. I.

\bibitem[\protect\citeauthoryear{Johnson,~G.~G.}{1937}]{johnsongg}
  Johnson, Guion Griffis.  \emph{Ante-Bellum North Carolina, A social History}.
  Chapel Hill, University of North Carolina Press, 1937.

\bibitem[\protect\citeauthoryear{Lee,~L}{1937}]{leel}
  Lee, Lawrence.  \emph{The Lower Cape Fear in Colonial Days}.
  Chapel Hill, University of North Carolina Press, 1937.

\bibitem[\protect\citeauthoryear{Lefler,~H.~T.}{1956}]{leflerht}
  Lefler, Hugh Talmadge.  \emph{History of North Carolina}, 4 volumes.  New
  York, Lewis Publishing Company, 1956.

\bibitem[\protect\citeauthoryear{Oates,~J.~A.}{1950}]{oatesja}
  Oates, John A.  \emph{The Story of Fayetteville and the Upper Cape Fear}.
  Charlotte, The Dowd press, 1950.

\bibitem[\protect\citeauthoryear{Schaw,~J.}{1939}]{schawj}
  Schaw, Janet.  \emph{Journal of a Lady of Quality}.  New Haven, Yale
  University Press, 1939.  Spartanburg, S.~C.~reprint.

\section{Traditional Sources}

\bibitem[\protect\citeauthoryear{Shaw, G.}{}]{shawg}
  Shaw, Graham, Atkinson, N.~C., native of Sampson County, N.~C.

\end{thebibliography}

\end{document}
